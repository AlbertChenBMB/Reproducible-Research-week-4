% Options for packages loaded elsewhere
\PassOptionsToPackage{unicode}{hyperref}
\PassOptionsToPackage{hyphens}{url}
%
\documentclass[
]{article}
\usepackage{lmodern}
\usepackage{amssymb,amsmath}
\usepackage{ifxetex,ifluatex}
\ifnum 0\ifxetex 1\fi\ifluatex 1\fi=0 % if pdftex
  \usepackage[T1]{fontenc}
  \usepackage[utf8]{inputenc}
  \usepackage{textcomp} % provide euro and other symbols
\else % if luatex or xetex
  \usepackage{unicode-math}
  \defaultfontfeatures{Scale=MatchLowercase}
  \defaultfontfeatures[\rmfamily]{Ligatures=TeX,Scale=1}
\fi
% Use upquote if available, for straight quotes in verbatim environments
\IfFileExists{upquote.sty}{\usepackage{upquote}}{}
\IfFileExists{microtype.sty}{% use microtype if available
  \usepackage[]{microtype}
  \UseMicrotypeSet[protrusion]{basicmath} % disable protrusion for tt fonts
}{}
\makeatletter
\@ifundefined{KOMAClassName}{% if non-KOMA class
  \IfFileExists{parskip.sty}{%
    \usepackage{parskip}
  }{% else
    \setlength{\parindent}{0pt}
    \setlength{\parskip}{6pt plus 2pt minus 1pt}}
}{% if KOMA class
  \KOMAoptions{parskip=half}}
\makeatother
\usepackage{xcolor}
\IfFileExists{xurl.sty}{\usepackage{xurl}}{} % add URL line breaks if available
\IfFileExists{bookmark.sty}{\usepackage{bookmark}}{\usepackage{hyperref}}
\hypersetup{
  pdftitle={Storm data analysis},
  pdfauthor={Shin-Fu Chen},
  hidelinks,
  pdfcreator={LaTeX via pandoc}}
\urlstyle{same} % disable monospaced font for URLs
\usepackage[margin=1in]{geometry}
\usepackage{color}
\usepackage{fancyvrb}
\newcommand{\VerbBar}{|}
\newcommand{\VERB}{\Verb[commandchars=\\\{\}]}
\DefineVerbatimEnvironment{Highlighting}{Verbatim}{commandchars=\\\{\}}
% Add ',fontsize=\small' for more characters per line
\usepackage{framed}
\definecolor{shadecolor}{RGB}{248,248,248}
\newenvironment{Shaded}{\begin{snugshade}}{\end{snugshade}}
\newcommand{\AlertTok}[1]{\textcolor[rgb]{0.94,0.16,0.16}{#1}}
\newcommand{\AnnotationTok}[1]{\textcolor[rgb]{0.56,0.35,0.01}{\textbf{\textit{#1}}}}
\newcommand{\AttributeTok}[1]{\textcolor[rgb]{0.77,0.63,0.00}{#1}}
\newcommand{\BaseNTok}[1]{\textcolor[rgb]{0.00,0.00,0.81}{#1}}
\newcommand{\BuiltInTok}[1]{#1}
\newcommand{\CharTok}[1]{\textcolor[rgb]{0.31,0.60,0.02}{#1}}
\newcommand{\CommentTok}[1]{\textcolor[rgb]{0.56,0.35,0.01}{\textit{#1}}}
\newcommand{\CommentVarTok}[1]{\textcolor[rgb]{0.56,0.35,0.01}{\textbf{\textit{#1}}}}
\newcommand{\ConstantTok}[1]{\textcolor[rgb]{0.00,0.00,0.00}{#1}}
\newcommand{\ControlFlowTok}[1]{\textcolor[rgb]{0.13,0.29,0.53}{\textbf{#1}}}
\newcommand{\DataTypeTok}[1]{\textcolor[rgb]{0.13,0.29,0.53}{#1}}
\newcommand{\DecValTok}[1]{\textcolor[rgb]{0.00,0.00,0.81}{#1}}
\newcommand{\DocumentationTok}[1]{\textcolor[rgb]{0.56,0.35,0.01}{\textbf{\textit{#1}}}}
\newcommand{\ErrorTok}[1]{\textcolor[rgb]{0.64,0.00,0.00}{\textbf{#1}}}
\newcommand{\ExtensionTok}[1]{#1}
\newcommand{\FloatTok}[1]{\textcolor[rgb]{0.00,0.00,0.81}{#1}}
\newcommand{\FunctionTok}[1]{\textcolor[rgb]{0.00,0.00,0.00}{#1}}
\newcommand{\ImportTok}[1]{#1}
\newcommand{\InformationTok}[1]{\textcolor[rgb]{0.56,0.35,0.01}{\textbf{\textit{#1}}}}
\newcommand{\KeywordTok}[1]{\textcolor[rgb]{0.13,0.29,0.53}{\textbf{#1}}}
\newcommand{\NormalTok}[1]{#1}
\newcommand{\OperatorTok}[1]{\textcolor[rgb]{0.81,0.36,0.00}{\textbf{#1}}}
\newcommand{\OtherTok}[1]{\textcolor[rgb]{0.56,0.35,0.01}{#1}}
\newcommand{\PreprocessorTok}[1]{\textcolor[rgb]{0.56,0.35,0.01}{\textit{#1}}}
\newcommand{\RegionMarkerTok}[1]{#1}
\newcommand{\SpecialCharTok}[1]{\textcolor[rgb]{0.00,0.00,0.00}{#1}}
\newcommand{\SpecialStringTok}[1]{\textcolor[rgb]{0.31,0.60,0.02}{#1}}
\newcommand{\StringTok}[1]{\textcolor[rgb]{0.31,0.60,0.02}{#1}}
\newcommand{\VariableTok}[1]{\textcolor[rgb]{0.00,0.00,0.00}{#1}}
\newcommand{\VerbatimStringTok}[1]{\textcolor[rgb]{0.31,0.60,0.02}{#1}}
\newcommand{\WarningTok}[1]{\textcolor[rgb]{0.56,0.35,0.01}{\textbf{\textit{#1}}}}
\usepackage{graphicx,grffile}
\makeatletter
\def\maxwidth{\ifdim\Gin@nat@width>\linewidth\linewidth\else\Gin@nat@width\fi}
\def\maxheight{\ifdim\Gin@nat@height>\textheight\textheight\else\Gin@nat@height\fi}
\makeatother
% Scale images if necessary, so that they will not overflow the page
% margins by default, and it is still possible to overwrite the defaults
% using explicit options in \includegraphics[width, height, ...]{}
\setkeys{Gin}{width=\maxwidth,height=\maxheight,keepaspectratio}
% Set default figure placement to htbp
\makeatletter
\def\fps@figure{htbp}
\makeatother
\setlength{\emergencystretch}{3em} % prevent overfull lines
\providecommand{\tightlist}{%
  \setlength{\itemsep}{0pt}\setlength{\parskip}{0pt}}
\setcounter{secnumdepth}{-\maxdimen} % remove section numbering

\title{Storm data analysis}
\author{Shin-Fu Chen}
\date{2020/02/01}

\begin{document}
\maketitle

\#\#require package

\begin{Shaded}
\begin{Highlighting}[]
\KeywordTok{library}\NormalTok{(dplyr)}
\end{Highlighting}
\end{Shaded}

\begin{verbatim}
## 
## Attaching package: 'dplyr'
\end{verbatim}

\begin{verbatim}
## The following objects are masked from 'package:stats':
## 
##     filter, lag
\end{verbatim}

\begin{verbatim}
## The following objects are masked from 'package:base':
## 
##     intersect, setdiff, setequal, union
\end{verbatim}

\begin{Shaded}
\begin{Highlighting}[]
\KeywordTok{library}\NormalTok{(readr)}
\KeywordTok{library}\NormalTok{(ggplot2)}
\end{Highlighting}
\end{Shaded}

\hypertarget{storm-data-analysis}{%
\subsection{Storm data analysis}\label{storm-data-analysis}}

This is an analysis report of storm data. There are two questions we
want to analyze, they are:

\begin{enumerate}
\def\labelenumi{\arabic{enumi}.}
\item
  Across the United States, which types of events are most harmful with
  respect to population health?
\item
  Across the United States, which types of events have the greatest
  economic consequences?
\end{enumerate}

\begin{center}\rule{0.5\linewidth}{0.5pt}\end{center}

In this report, there are four parts of chapter, first is Synopsis,
second is Data Processing, third is Results, and fourth is summary. ***
\#\# Synopsis

\hypertarget{data-processing}{%
\subsection{Data Processing}\label{data-processing}}

Firstly, we read the data from the original bz2 file and access it into
a dataset named ``rawStorm''.

\begin{Shaded}
\begin{Highlighting}[]
\NormalTok{rawStorm <-}\StringTok{ }\KeywordTok{tbl_df}\NormalTok{(}\KeywordTok{read_csv}\NormalTok{(}\StringTok{"repdata_data_StormData.csv.bz2"}\NormalTok{))}
\end{Highlighting}
\end{Shaded}

\begin{verbatim}
## Parsed with column specification:
## cols(
##   .default = col_double(),
##   BGN_DATE = col_character(),
##   BGN_TIME = col_character(),
##   TIME_ZONE = col_character(),
##   COUNTYNAME = col_character(),
##   STATE = col_character(),
##   EVTYPE = col_character(),
##   BGN_AZI = col_logical(),
##   BGN_LOCATI = col_logical(),
##   END_DATE = col_logical(),
##   END_TIME = col_logical(),
##   COUNTYENDN = col_logical(),
##   END_AZI = col_logical(),
##   END_LOCATI = col_logical(),
##   PROPDMGEXP = col_character(),
##   CROPDMGEXP = col_logical(),
##   WFO = col_logical(),
##   STATEOFFIC = col_logical(),
##   ZONENAMES = col_logical(),
##   REMARKS = col_logical()
## )
\end{verbatim}

\begin{verbatim}
## See spec(...) for full column specifications.
\end{verbatim}

\begin{verbatim}
## Warning: 5255570 parsing failures.
##  row col           expected actual                             file
## 1671 WFO 1/0/T/F/TRUE/FALSE     NG 'repdata_data_StormData.csv.bz2'
## 1673 WFO 1/0/T/F/TRUE/FALSE     NG 'repdata_data_StormData.csv.bz2'
## 1674 WFO 1/0/T/F/TRUE/FALSE     NG 'repdata_data_StormData.csv.bz2'
## 1675 WFO 1/0/T/F/TRUE/FALSE     NG 'repdata_data_StormData.csv.bz2'
## 1678 WFO 1/0/T/F/TRUE/FALSE     NG 'repdata_data_StormData.csv.bz2'
## .... ... .................. ...... ................................
## See problems(...) for more details.
\end{verbatim}

\begin{Shaded}
\begin{Highlighting}[]
\KeywordTok{str}\NormalTok{(rawStorm)}
\end{Highlighting}
\end{Shaded}

\begin{verbatim}
## Classes 'tbl_df', 'tbl' and 'data.frame':    902297 obs. of  37 variables:
##  $ STATE__   : num  1 1 1 1 1 1 1 1 1 1 ...
##  $ BGN_DATE  : chr  "4/18/1950 0:00:00" "4/18/1950 0:00:00" "2/20/1951 0:00:00" "6/8/1951 0:00:00" ...
##  $ BGN_TIME  : chr  "0130" "0145" "1600" "0900" ...
##  $ TIME_ZONE : chr  "CST" "CST" "CST" "CST" ...
##  $ COUNTY    : num  97 3 57 89 43 77 9 123 125 57 ...
##  $ COUNTYNAME: chr  "MOBILE" "BALDWIN" "FAYETTE" "MADISON" ...
##  $ STATE     : chr  "AL" "AL" "AL" "AL" ...
##  $ EVTYPE    : chr  "TORNADO" "TORNADO" "TORNADO" "TORNADO" ...
##  $ BGN_RANGE : num  0 0 0 0 0 0 0 0 0 0 ...
##  $ BGN_AZI   : logi  NA NA NA NA NA NA ...
##  $ BGN_LOCATI: logi  NA NA NA NA NA NA ...
##  $ END_DATE  : logi  NA NA NA NA NA NA ...
##  $ END_TIME  : logi  NA NA NA NA NA NA ...
##  $ COUNTY_END: num  0 0 0 0 0 0 0 0 0 0 ...
##  $ COUNTYENDN: logi  NA NA NA NA NA NA ...
##  $ END_RANGE : num  0 0 0 0 0 0 0 0 0 0 ...
##  $ END_AZI   : logi  NA NA NA NA NA NA ...
##  $ END_LOCATI: logi  NA NA NA NA NA NA ...
##  $ LENGTH    : num  14 2 0.1 0 0 1.5 1.5 0 3.3 2.3 ...
##  $ WIDTH     : num  100 150 123 100 150 177 33 33 100 100 ...
##  $ F         : num  3 2 2 2 2 2 2 1 3 3 ...
##  $ MAG       : num  0 0 0 0 0 0 0 0 0 0 ...
##  $ FATALITIES: num  0 0 0 0 0 0 0 0 1 0 ...
##  $ INJURIES  : num  15 0 2 2 2 6 1 0 14 0 ...
##  $ PROPDMG   : num  25 2.5 25 2.5 2.5 2.5 2.5 2.5 25 25 ...
##  $ PROPDMGEXP: chr  "K" "K" "K" "K" ...
##  $ CROPDMG   : num  0 0 0 0 0 0 0 0 0 0 ...
##  $ CROPDMGEXP: logi  NA NA NA NA NA NA ...
##  $ WFO       : logi  NA NA NA NA NA NA ...
##  $ STATEOFFIC: logi  NA NA NA NA NA NA ...
##  $ ZONENAMES : logi  NA NA NA NA NA NA ...
##  $ LATITUDE  : num  3040 3042 3340 3458 3412 ...
##  $ LONGITUDE : num  8812 8755 8742 8626 8642 ...
##  $ LATITUDE_E: num  3051 0 0 0 0 ...
##  $ LONGITUDE_: num  8806 0 0 0 0 ...
##  $ REMARKS   : logi  NA NA NA NA NA NA ...
##  $ REFNUM    : num  1 2 3 4 5 6 7 8 9 10 ...
##  - attr(*, "problems")=Classes 'tbl_df', 'tbl' and 'data.frame': 5255570 obs. of  5 variables:
##   ..$ row     : int  1671 1673 1674 1675 1678 1679 1680 1681 1682 1683 ...
##   ..$ col     : chr  "WFO" "WFO" "WFO" "WFO" ...
##   ..$ expected: chr  "1/0/T/F/TRUE/FALSE" "1/0/T/F/TRUE/FALSE" "1/0/T/F/TRUE/FALSE" "1/0/T/F/TRUE/FALSE" ...
##   ..$ actual  : chr  "NG" "NG" "NG" "NG" ...
##   ..$ file    : chr  "'repdata_data_StormData.csv.bz2'" "'repdata_data_StormData.csv.bz2'" "'repdata_data_StormData.csv.bz2'" "'repdata_data_StormData.csv.bz2'" ...
##  - attr(*, "spec")=
##   .. cols(
##   ..   STATE__ = col_double(),
##   ..   BGN_DATE = col_character(),
##   ..   BGN_TIME = col_character(),
##   ..   TIME_ZONE = col_character(),
##   ..   COUNTY = col_double(),
##   ..   COUNTYNAME = col_character(),
##   ..   STATE = col_character(),
##   ..   EVTYPE = col_character(),
##   ..   BGN_RANGE = col_double(),
##   ..   BGN_AZI = col_logical(),
##   ..   BGN_LOCATI = col_logical(),
##   ..   END_DATE = col_logical(),
##   ..   END_TIME = col_logical(),
##   ..   COUNTY_END = col_double(),
##   ..   COUNTYENDN = col_logical(),
##   ..   END_RANGE = col_double(),
##   ..   END_AZI = col_logical(),
##   ..   END_LOCATI = col_logical(),
##   ..   LENGTH = col_double(),
##   ..   WIDTH = col_double(),
##   ..   F = col_double(),
##   ..   MAG = col_double(),
##   ..   FATALITIES = col_double(),
##   ..   INJURIES = col_double(),
##   ..   PROPDMG = col_double(),
##   ..   PROPDMGEXP = col_character(),
##   ..   CROPDMG = col_double(),
##   ..   CROPDMGEXP = col_logical(),
##   ..   WFO = col_logical(),
##   ..   STATEOFFIC = col_logical(),
##   ..   ZONENAMES = col_logical(),
##   ..   LATITUDE = col_double(),
##   ..   LONGITUDE = col_double(),
##   ..   LATITUDE_E = col_double(),
##   ..   LONGITUDE_ = col_double(),
##   ..   REMARKS = col_logical(),
##   ..   REFNUM = col_double()
##   .. )
\end{verbatim}

The rawStorm data containt with 37 variables and 1773320 objects.
However, we want to analysis which event is most harmful with human, so
we only need object with event.

\begin{Shaded}
\begin{Highlighting}[]
\NormalTok{rawStorm<-}\StringTok{ }\KeywordTok{filter}\NormalTok{(rawStorm,EVTYPE}\OperatorTok{!=}\StringTok{""}\NormalTok{,EVTYPE}\OperatorTok{!=}\StringTok{" "}\NormalTok{,}\OperatorTok{!}\KeywordTok{is.na}\NormalTok{(EVTYPE))}
\NormalTok{rawStorm }\OperatorTok\StringTok{ }\KeywordTok{group_by}\NormalTok{(EVTYPE)}
\end{Highlighting}
\end{Shaded}

\begin{verbatim}
## # A tibble: 902,297 x 37
## # Groups:   EVTYPE [977]
##    STATE__ BGN_DATE BGN_TIME TIME_ZONE COUNTY COUNTYNAME STATE EVTYPE BGN_RANGE
##      <dbl> <chr>    <chr>    <chr>      <dbl> <chr>      <chr> <chr>      <dbl>
##  1       1 4/18/19~ 0130     CST           97 MOBILE     AL    TORNA~         0
##  2       1 4/18/19~ 0145     CST            3 BALDWIN    AL    TORNA~         0
##  3       1 2/20/19~ 1600     CST           57 FAYETTE    AL    TORNA~         0
##  4       1 6/8/195~ 0900     CST           89 MADISON    AL    TORNA~         0
##  5       1 11/15/1~ 1500     CST           43 CULLMAN    AL    TORNA~         0
##  6       1 11/15/1~ 2000     CST           77 LAUDERDALE AL    TORNA~         0
##  7       1 11/16/1~ 0100     CST            9 BLOUNT     AL    TORNA~         0
##  8       1 1/22/19~ 0900     CST          123 TALLAPOOSA AL    TORNA~         0
##  9       1 2/13/19~ 2000     CST          125 TUSCALOOSA AL    TORNA~         0
## 10       1 2/13/19~ 2000     CST           57 FAYETTE    AL    TORNA~         0
## # ... with 902,287 more rows, and 28 more variables: BGN_AZI <lgl>,
## #   BGN_LOCATI <lgl>, END_DATE <lgl>, END_TIME <lgl>, COUNTY_END <dbl>,
## #   COUNTYENDN <lgl>, END_RANGE <dbl>, END_AZI <lgl>, END_LOCATI <lgl>,
## #   LENGTH <dbl>, WIDTH <dbl>, F <dbl>, MAG <dbl>, FATALITIES <dbl>,
## #   INJURIES <dbl>, PROPDMG <dbl>, PROPDMGEXP <chr>, CROPDMG <dbl>,
## #   CROPDMGEXP <lgl>, WFO <lgl>, STATEOFFIC <lgl>, ZONENAMES <lgl>,
## #   LATITUDE <dbl>, LONGITUDE <dbl>, LATITUDE_E <dbl>, LONGITUDE_ <dbl>,
## #   REMARKS <lgl>, REFNUM <dbl>
\end{verbatim}

Because we have two different questions to analyze, so we split the data
into two data sets which contain with different variables. to find out
most harmful with respect to population health, we need to analysis
fatalities and injuries. we summary total fatalities and total injures
by each event, and show the top 3 events.

\begin{Shaded}
\begin{Highlighting}[]
\NormalTok{harmful_with_health<-}\KeywordTok{select}\NormalTok{(rawStorm ,EVTYPE,FATALITIES,INJURIES)}\OperatorTok
\StringTok{                      }\KeywordTok{group_by}\NormalTok{(EVTYPE)}\OperatorTok
\StringTok{                      }\KeywordTok{summarize}\NormalTok{(}\DataTypeTok{total_fatalities=}\KeywordTok{sum}\NormalTok{(FATALITIES),}\DataTypeTok{total_injuries=}\KeywordTok{sum}\NormalTok{(INJURIES))}\OperatorTok
\StringTok{                      }\KeywordTok{arrange}\NormalTok{(}\KeywordTok{desc}\NormalTok{(total_fatalities),}\KeywordTok{desc}\NormalTok{(total_injuries))}\OperatorTok
\StringTok{                      }\KeywordTok{head}\NormalTok{(}\DecValTok{3}\NormalTok{)}
\CommentTok{# cost_of_economic<-rawStorm [,c("X.EVTYPE.", "X.PROPDMG.","X.CROPDMG.")]}
\NormalTok{harmful_with_health}
\end{Highlighting}
\end{Shaded}

\begin{verbatim}
## # A tibble: 3 x 3
##   EVTYPE         total_fatalities total_injuries
##   <chr>                     <dbl>          <dbl>
## 1 TORNADO                    5633          91346
## 2 EXCESSIVE HEAT             1903           6525
## 3 FLASH FLOOD                 978           1777
\end{verbatim}

to show the comparision of total fatalities and total injuries, we apply
bar plot as follow

\includegraphics{storm-data-analysis_files/figure-latex/unnamed-chunk-5-1.pdf}
\includegraphics{storm-data-analysis_files/figure-latex/unnamed-chunk-5-2.pdf}
to find out the greatest economic consequences, we need to analysis
property damage and crop damage. however, the variable ``PROPDMGEXP''
and ``CROPDMGEXP'' are\ldots..

\begin{Shaded}
\begin{Highlighting}[]
\NormalTok{cost_of_economic<-}\KeywordTok{select}\NormalTok{(rawStorm ,EVTYPE,PROPDMG,PROPDMGEXP,CROPDMG,CROPDMGEXP)}\OperatorTok
\StringTok{                      }\KeywordTok{mutate}\NormalTok{(}\DataTypeTok{prop_exp=}\KeywordTok{ifelse}\NormalTok{(PROPDMGEXP}\OperatorTok{==}\StringTok{"B"}\OperatorTok{|}\NormalTok{PROPDMGEXP}\OperatorTok{==}\StringTok{"b"}\NormalTok{,}\DecValTok{1000000000}\NormalTok{,}
                                         \KeywordTok{ifelse}\NormalTok{(PROPDMGEXP}\OperatorTok{==}\StringTok{"M"}\OperatorTok{|}\NormalTok{PROPDMGEXP}\OperatorTok{==}\StringTok{"m"}\NormalTok{,}\DecValTok{1000000}\NormalTok{,}
                                                \KeywordTok{ifelse}\NormalTok{(PROPDMGEXP}\OperatorTok{==}\StringTok{"K"}\OperatorTok{|}\NormalTok{PROPDMGEXP}\OperatorTok{==}\StringTok{"k"}\NormalTok{,}\DecValTok{1000}\NormalTok{,}
                                                       \KeywordTok{ifelse}\NormalTok{(PROPDMGEXP}\OperatorTok{==}\StringTok{"H"}\OperatorTok{|}\NormalTok{PROPDMGEXP}\OperatorTok{==}\StringTok{"h"}\NormalTok{,}\DecValTok{100}\NormalTok{,}
                                                              \DecValTok{1}\NormalTok{)))),}
                         \DataTypeTok{crop_exp=}\KeywordTok{ifelse}\NormalTok{(CROPDMGEXP}\OperatorTok{==}\StringTok{"B"}\OperatorTok{|}\NormalTok{CROPDMGEXP}\OperatorTok{==}\StringTok{"b"}\NormalTok{,}\DecValTok{1000000000}\NormalTok{,}
                                         \KeywordTok{ifelse}\NormalTok{(CROPDMGEXP}\OperatorTok{==}\StringTok{"M"}\OperatorTok{|}\NormalTok{CROPDMGEXP}\OperatorTok{==}\StringTok{"m"}\NormalTok{,}\DecValTok{1000000}\NormalTok{,}
                                                \KeywordTok{ifelse}\NormalTok{(CROPDMGEXP}\OperatorTok{==}\StringTok{"K"}\OperatorTok{|}\NormalTok{CROPDMGEXP}\OperatorTok{==}\StringTok{"k"}\NormalTok{,}\DecValTok{1000}\NormalTok{,}
                                                       \KeywordTok{ifelse}\NormalTok{(CROPDMGEXP}\OperatorTok{==}\StringTok{"H"}\OperatorTok{|}\NormalTok{CROPDMGEXP}\OperatorTok{==}\StringTok{"h"}\NormalTok{,}\DecValTok{100}\NormalTok{,}
                                                              \DecValTok{1}\NormalTok{)))),}
                         \DataTypeTok{prop_dm=}\NormalTok{PROPDMG}\OperatorTok{*}\NormalTok{prop_exp,}\DataTypeTok{crop_dm=}\NormalTok{CROPDMG}\OperatorTok{*}\NormalTok{crop_exp)}\OperatorTok
\StringTok{                          }\KeywordTok{group_by}\NormalTok{(EVTYPE)}\OperatorTok
\StringTok{                          }\KeywordTok{summarize}\NormalTok{(}\DataTypeTok{total_prop_dm=}\KeywordTok{sum}\NormalTok{(prop_dm,}\DataTypeTok{na.rm =}\NormalTok{ T),}\DataTypeTok{total_crop_dm=}\KeywordTok{sum}\NormalTok{(crop_dm,}\DataTypeTok{na.rm =}\NormalTok{ T))}\OperatorTok
\StringTok{                      }\KeywordTok{arrange}\NormalTok{(}\KeywordTok{desc}\NormalTok{(total_crop_dm),}\KeywordTok{desc}\NormalTok{(total_prop_dm))}\OperatorTok
\StringTok{                      }\KeywordTok{head}\NormalTok{(}\DecValTok{5}\NormalTok{)}

\NormalTok{cost_of_economic}
\end{Highlighting}
\end{Shaded}

\begin{verbatim}
## # A tibble: 5 x 3
##   EVTYPE             total_prop_dm total_crop_dm
##   <chr>                      <dbl>         <dbl>
## 1 TORNADO             56937160776.           160
## 2 THUNDERSTORM WINDS   1735960802.            80
## 3 HAIL                15732267486.            20
## 4 FLOOD              144657709800              0
## 5 HURRICANE/TYPHOON   69305840000              0
\end{verbatim}

\hypertarget{including-plots}{%
\subsection{Including Plots}\label{including-plots}}

\end{document}
